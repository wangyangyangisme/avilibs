\documentclass[12pt, a4paper]{article}

\usepackage{verbatim}

\font\sevenrm=cmr7
\font\ninerm=cmr9
\font\bigfont=cmb12 scaled\magstep2
\font\bigsymbol=cmsy10 scaled\magstep2
\font\kapitalakia=cmcsc10 scaled\magstephalf

\def\comment#1 {\vskip 9pt {\bigsymbol \char'051} {\ninerm #1 } \vskip 9pt}
% \def\comment#1 {}

\def\superscript#1 {\hbox{\sevenrm \raise .5ex \hbox{#1}}}

\begin{document}

\newcommand{\libavi}[0]{\emph{libavi}}
\newcommand{\AVI}[0]{{\kapitalakia AVI}}
\newcommand{\mymail}[0]{\texttt{<ptsant@otenet.gr>}}
\newcommand{\libversion}[0]{{\bf 0.69}}

\title{\emph{libavi}: a library that handles AVI files}
\author{Petros K. Tsantoulis}
\date{11/01/2002}

\maketitle
\vskip 40ex
\hfil\mymail{}\hfil

\hfil\verb+http://avilibs.sourceforge.net+\hfil
\vskip 8ex
\hfil{Version \libversion{}}\hfil
\newpage
\tableofcontents
\newpage

\section{Introduction}
This library is being released to the general public under the terms of the 
popular GPL version 2. This is a C library of functions that tries to facilitate
the manipulation of \AVI{} files. It requires \emph{libriff} to build succesfully.
This library does {\bf not decode frame content}.
It never will. This library is a {\bf file manipulation library}. It will happily
read or write frames (chunks) according to the \AVI{} structure but it will not
encode or decode their content. This is a seperate task that will be implemented
either in the calling application or in a third library that I plan to develop in
the future.

The overall design of \libavi{} is quite simple. There is a mechanism for reading
and a seperate mechanism for writing. Other than being aware of the \AVI{} file 
structure this library also offers a semi-automatic \emph{stream support} mode that 
greatly simplifies reading and writing by handling streams as seperate entities. 
If you wish to read/write \AVI{} files with 
\libavi{} you have two choices\footnote{You may also use \emph{libriff} for maximum
control over the file I/O, but this means that your application must be able to parse
the \AVI{} structure and generate valid \AVI{} files. This is not recommended.}:
\begin{description}
\item[AVI file I/O]: You can use \libavi{} to read/write the headers and then read/write the frames 
and de-multiplex them in your application. This mode offers some ``\AVI{}-awareness'' in the sense that
you won't have to parse the header and index structure. However, you get full control
over the {\it movi\/} list and its contents.
\item[Stream support]: You can use \libavi{} to seek and read frames from individual streams.
In this semi-automatic mode the library handles de-multiplexing
and seeking (and, to some extent stream synchronization) and all you have to do is handle
the frame contents yourself. Each stream is a seperate entity.
\end{description}

I decided to build this manual in the form of a tutorial because learning by example
is much easier and \libavi{} is actually oriented to a specific purpose (unlike
\emph{libriff} that is an all-purpose tool and requires an all-purpose manual).
For questions of a more technical nature you should consult the header and source files. 
They are quite thoroughly documented with verbose comments, including function
input, output and side effects.

%\subsection*{Limitations}
%\subsection*{Name conventions}

\section{Current Status (version \libversion{})}
Unfortunately, \libavi{} is not yet complete. I'd say that quite a lot remains to be
done. This section has been added to warn you about what works and what does \emph{not}
work in the current version. I decided to start writing this tutorial before 
completing \libavi{} because the current functionality is adequate for many simple
tasks and quite possibly some people would like to start working with it, or \emph{on}
it.

The current version (\libversion{}) does not implement stream support for writing. Writing
is much easier than reading because writing need only produce a valid \AVI{} file, while
reading should properly handle all valid \AVI{} files. Writing does work but 
without any kind of automation, e.g. the index is not built automatically, the streams
are not automatically synchronized. You should be able to get valid \AVI{} files
without much trouble but writing will definitely be automated in the future with the
addition of stream support. 

Reading can properly parse the header structure, including exotic chunks such as
\verb+strn+.
Stream support for reading 
currently allows only seeking by one frame either forward or backward. It will work
with or without an index chunk. The current version is also capable of building
an index chunk for a file with a corrupt index or without an index. Please note that
building an index for a large \AVI{} file can be time consuming due to heavy disk
usage. This is not a bug in \libavi{}.

The major problem with the current version and one that must be solved in the near
future is the timing of audio streams. I'm unable to properly time audio streams,
especially {\kapitalakia VBR} ones. If someone knows how to do this or can point
me to an authoritative source please do contact me. Note that the library is
able to return audio frames in their proper order and can demultiplex them
without problem. The problem is that the \verb+stream_time[]+ field in the \verb+avi_handle+
structure is currently not reliable for audio frames.

Another minor nuisance is the lack of broader seeking abilities such as seeking
to some time point, seeking to the Nth frame etc. This can be circumvented very
easily in the calling application by proper usage of the existing functions, but
with a small performance hit. These functions will be added in the future.

Note that by design \libavi{} will handle multiple streams, up to the maximum
theoretical limit of 100 streams. The stream number (00, 01 etc) serves as
the only identity that distinguishes its related frames. This means that
frames 00dc, 00wb, 00pc will all be reported as belonging to stream 00 
regardless of stream type. A proper \AVI{} file should not cause any problems.
This is not a bug in \libavi{}.

\section{Reading AVI files}

\subsection{Opening the file}
The following code will open the file and associate it with the \verb+avi_handle+ 
structure for future reference. It will parse the header and index.
Note that, like \emph{libriff}, all function names in \libavi{} are prefixed
with {\kapitalakia AVR} if they are related to reading and {\kapitalakia AVW}
if they are related to writing. A prepended underscore (\verb+_+) shows that
the function is intended for internal use. You may use these so-called internal
functions if you need to but they are not really the front end of the library.
This small piece of code prints out a lot of interesting information.
Try it out and see. It is included under the filename \verb+example1.c+

\verbatiminput{example1.c}

\subsection{Seeking a little bit}
Now we will try to move a little inside the file. But first you need to understand a 
few things about the way \libavi{} sees the file. The library always keeps N+1
file descriptors open\footnote{This
requires a theoretical maximum of 101 simultaneously open file descriptors for 
every \AVI{} file but this is extremely unlikely to occur in everyday practice.
Normally 3 descriptors will be open for every file. The Linux kernel can easily 
handle 1024 open files.}
for every \AVI{} file, where N is the number of streams.
The library actually associates N \verb+riff_handle+ structures, one for each stream,
with the file and every one of them set to ``linear view'', i.e. without list 
information from \emph{libriff}.
The remaining \verb+riff_handle+ is the main handle that is used to parse the header and
index. This handle---accessible through \verb+avi_handle->riff_file+---is not set
to linear view and cannot be used for raw file seeking because it will upset list information.

If you decide to do your own I/O all you need to do is use the traditional \emph{libriff}
functions such as \verb+RFRget_chunk()+ and \verb+RFRskip_chunk()+ on the main
riff handle (\verb+avi_handle->riff_file+). You may use raw seeking (\verb+fseek()+)
if you set it to linear view or you may choose to use the functions from 
\verb+riffseek.h+.

If on the other hand you use the stream support that the library offers, you only deal
with streams numbered from 0 to \verb+avi_handle->total_streams-1+ and
you don't care about the underlying file.

In the next simple example we will use stream support to read 3 frames from stream 0, 
skip 2 frames in stream 0, read 2 frames from stream 1 and then go back 1 frame in stream 0.
This example expands example 1 and is also available under the filename
\verb+example2.c+. All seeking is done via the index chunk provided, so make
sure that the \AVI{} file you specify on the command line has an index chunk
and at least two streams.

\verbatiminput{example2.c}

Note that getting
the frame via \verb+AVRget_frame_idx+ does not advance the index pointer.
If you call it again you will get the same frame, again.

\subsection{Seeking without the index chunk}
For those that wish to seek without referring to the index chunk the function
\verb+AVRnext_frame()+ is able to find the next frame that belongs to the
requested stream. You may then use the function \verb+AVRget_frame()+ to read
the frame data for the requested stream. Note that getting the frame \emph{moves} 
the file pointer and therefore you may not need to call \verb+AVRnext_frame()+ 
to move to the next frame. A trivial example of this can be found in the
test code, inside the test directory. I will not quote this code here
because it has not been nicely formatted for postscript output.

%\verbatiminput{no-index.c}

%Simply use it instead of \verb+AVRnext_frame_idx()+. 
%The above code reads 20 frames from the file, so if the file has 2 streams you
%are likely to get 10 frames from each. 
Seeking
backwards is \emph{not} supported without an index chunk because the 
{\kapitalakia RIFF} file format is not well suited to moving backwards. 
I \emph{strongly} suggest
that you use the index unless you have a very special reason to avoid it.
The functions that do not use the index have not been extensively tested, but they should work
without much trouble.

\subsection{Building the index chunk}
In cases where the index chunk is corrupt or missing you can use the library
function \verb+_AVRbuild_idx()+ to create an index in memory. This is time
consuming because it scans through the whole file. You can then use
all the functions that require an index such as \verb+AVRnext_frame_idx()+.

%\section{Writing \AVI{} files}

\section{Epilogue}
Well, that's all for a few months. If this version does not contain major
bugs it will not be updated soon. I will be extremely busy in the following
months and this project is likely to freeze. I'll propably catch up sometime
in the spring. Have fun!

\end{document}

