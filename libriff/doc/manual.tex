
\documentclass[12pt, a4paper]{article}


\font\sevenrm=cmr7
\font\ninerm=cmr9
\font\bigfont=cmb12 scaled\magstep2
\font\bigsymbol=cmsy10 scaled\magstep2

\def\comment#1 {\vskip 9pt {\bigsymbol \char'051} {\ninerm #1 } \vskip 9pt}
% \def\comment#1 {}

\def\superscript#1 {\hbox{\sevenrm \raise .5ex \hbox{#1}}}

\begin{document}

\newcommand{\libriff}[0]{\emph{libriff}}
\newcommand{\mymail}[0]{\texttt{<ptsant@otenet.gr>}}

\title{\emph{libriff}: a library that handles RIFF files}
\author{Petros K. Tsantoulis}
\date{20/08/2001}

\maketitle
\vskip 40ex
\hfil\mymail{}\hfil

\hfil\verb+http://avilibs.sourceforge.net+\hfil
\newpage
\tableofcontents
\newpage
\section{What's this?}
 This library tries to facilitate the handling of RIFF files, 
especially AVI (Audio/Video Interleave) files. It is part, 
indeed the core, of a larger project that will try to offer 
an easy to use solution for reading and writing of video 
and audio streams. I am releasing this library to the 
general public under the popular GNU GPL v.2 license in the
hope that it will be useful to free software developers. I 
reserve the right to change the licensing policy or restrict
it to GNU GPL v.2 if later versions of the GNU GPL are not 
satisfactory.

 This is a document that tries to give some information to future
developers and users of \libriff{}. It is not particularly thorough
or complete at the moment. This library and especially its documentation
are a work in progress. Feel free to bother me with e-mail 
at \texttt{<ptsant@otenet.gr>} for anything related to \libriff{}
(spam does {\bf not} count as related.)

\section{Installation}
Installation should be relatively painless. This library depends only
on the standard C library. The GNU C library version 2.2.3 has been used
and should be considered the library of reference. 

After uncompressing the sources in an appropriate directory (I must assume 
you have already done that) all you have to do is issue 
``make ; make install'' and the library will be installed under 
/usr/local/lib and /usr/local/include/libriff. Check the Makefile if 
you prefer some other destination directory. If you wish to rebuild the current 
manual from the .tex source you are going to need \LaTeX{} but this is not 
strictly necessary. Note that in recent versions a makefile for the mingw
developement environment in windows 32-bit has been included. This makefile,
named \verb+Makefile.w32+, was created by Simon Moser, along with the
code changes that allow you to build a windows dynamically linked 
library from the current code. I have not used \libriff{} in the win32
environment, but I believe that it works nicely (or, at least, not worse
than under linux).

\section{General Design}
\subsection{Object oriented}
 This library has been written in an implicitly object oriented manner.
This means that although C is not an ``object oriented'' language you are
expected to treat all internal data structures as ``private'' and manipulate
them solely with the aid of the accompanying functions, unless of
course you \emph{know} what you are doing (in which case, C gives you enough
rope to hang yourselves, if you so desire.) 

\subsection{Extending the library}
If further basic functionality is needed it can be implemented either on top
of the existing functions or on top of the existing data structures
(\texttt{{\bf struct} riff\_handle} .) If data structures are externally
modified you should take great care at all times to preserve data integrity.
Please note that you may \emph{read} values off the riff\_handle structure
but this structure may change in future versions while the function interface 
(if it exists) for the variable of interest should remain consistent. 
All functions do some sanity tests to ensure that the library maintains a correct 
and consistent internal environment. 

\subsection{The C language}
 The language C was chosen instead of C++ because it generally compiles better,
is compatible with C++ and results in smaller and faster executables. It can
be argued that C++ has several advantages, but this is a matter of personal opinion
and not subject of debate.

 Currently the library is faithful to the C89 coding standard, possibly with some
GNU C compiler extensions, but some C99 features may be implemented in the feature.
GCC v3.0 provides some C99 support and should be considered the reference 
C compiler. Older compilers should normally work without problems.

\section{Bugs, Features and Design Limitations} 
\subsection{Bugs}
No known bugs at the moment. Many of them probably exist but I haven't
seen them. Do contact me at \mymail{} if you see them.
\subsection{Features}
Several. I consider the most important to be :
\begin{description}
\item[Robustness.] This library contains internal checking code. It should
handle most errors gracefully and, most importantly, it allows you to easily
spot possible bugs.
\item[Portability.] This library aims towards maximum portability. It uses
standard C library functions and should compile on almost any platform.
Effort has been made to make it portable to big-endian machines but since
the author does not use such a machine for development the code has
been untested. 
\item[Freedom.] As in ``Freedom of Speech'', this library is released to the
general public under conditions that permit and encourage the sharing of
source code and ideas. This is a distinct advantage.
\end{description}
\subsection{Design Limitations.} 
There are some design limitations that may be troublesome in some cases.
I consider some of them to be (feel free to add your own) :
\begin{description}
\item[No internal buffering:] This library does not try to buffer data to
memory when reading or writing a file. This task is left to the C library
functions. I consider standard C library functions to be
sufficiently fast when coupled with an operating system that uses reasonable 
caching techniques.  I cannot supply you with measurements that prove my 
decision was correct but can speculate that the extra speed will not be 
worth the added complexity.
\item[Limited number of open lists:] This library will not handle RIFF
files that contain more than 1024 (an arbitrary number that can easily be
redefined in ``defines.h'') open, nested
lists. Any number of lists may be contained in the file but they may not
be nested deeper than 1024 at a time. I consider this to be a trivial 
limitation. Indeed it is not hard to add a dynamically growing data 
type such as a linked list that will solve this problem but this 
will be relatively slower. I really do not expect RIFF files with more 
than 1024 nested lists to exist. 
\item[Designed to handle AVI files:] The code was developed with AVI files
in mind and may not be appropriate for other types of RIFF files. The author
has made every effort to keep all functions generic and relatively extensible
but implicit assumptions could possibly limit the usefulness of this library
with other types of RIFF files. Please contact me to report such cases.
\item[No read/write mode:] The code will only handle reading
\emph{or} writing a file but not both at the same time. Editing a RIFF file
in place is quite difficult and cumbersome. You may achieve the same functionality
by using a read-only master file and a temporary output file. This allows for cleaner
code and more elegant design. 
\item[No indexing:] It is theoretically possible to build a mechanism for 
efficient traversal of the file by creating an index structure that
refers to the position of all (or perhaps the most recently visited) blocks
in the file. Its usability depends on the underlying type of RIFF and I 
think it should be part of specialized libraries.
\item[Single RIFF structure per file:] It is theoretically possible to add more RIFF
structures after the end of the first, but \libriff{} does not support reading
or writing this. You can use multiple files for that. After all, \libriff{}
properly handles 4GB RIFF files, which is the theoretical maximum, and also
a very large file size for many filesystems. 
\end{description}

\section{Usage}
 The main features of the library will be presented in a brief manner here. The
header and object source code should be considered the definite reference. 
You may be interested in reading the example code supplied with the library
sources.

\subsection{Basic RIFF structure}
 The fundamental blocks of RIFF files are \emph{chunks} and \emph{lists}.

\begin{verbatim}
CHUNK structure
<4 byte chunk type> (can be '00dc', 'strf' etc)
<4 byte chunk data size>
I: size counted from here
<chunk data>
\end{verbatim}

\begin{verbatim}
LIST structure
<4 byte word LIST ('L','I','S','T')> 
<4 byte list size>
I: size counted from here
<4 byte list type>
<blocks that belong to the list>
\end{verbatim}

Please note that all blocks begin on a 16-bit boundary. The recorded list or
chunk size only refers to the data that follows and does not take into 
account the header.

It is also apparent that the RIFF file is a forward singly-linked structure,
in the sense that even though it is relatively easy to seek forward, by jumping
to the next block, it is relatively difficult to seek backward. Indeed, an
efficient method of moving backwards has to be implemented by the calling 
program, perhaps in the form of an index structure. Currently this library
does not support such an indexing method. Some types of RIFF files, most
importantly AVI files, contain an appropriate index structure (the \verb+idx1+
chunk) that can be used to seek in the file. I believe that this kind
of indexing should be implemented in an outer layer that corresponds to the
RIFF file type of interest and not inside the generic core RIFF library.

\subsubsection{Important constants}
Some constants that are part of the RIFF file specification are:
\begin{verbatim}
/* RIFF file signature */
const uint32_t opRIFF=FOURCC('R','I','F','F'); 
/* LIST signature */
const uint32_t opLIST=FOURCC('L','I','S','T');
/* JUNK chunk signature */
const uint32_t opJUNK=FOURCC('J','U','N','K');
\end{verbatim}

These constants are defined inside the file ``riffcodes.h''. To
build your own see the file ``fourcc.h''.

\subsection{Opening files for reading}
 The library cannot handle a read/write mode. You can only read or write at
a file.

\subsubsection{RFRopen\_riff\_file()}
\begin{verbatim}
riff_handle * handle;

handle=RFRopen_riff_file(``filename'');

if (handle==0) 
	exit(-1); /* Could not open file */
else  
	...  /* File succesfully opened for reading */
\end{verbatim}

A four character code in the form of a \verb+uint32_t+ must 
be supplied to denote the type of RIFF file that is 
being created. For example, AVI files are of type
\verb+const uint32_t opAVI=0x20495641; /* = 'A','V','I',' ' */+

See the \verb+``fourcc.h''+ facilities for building \verb+uint32_t+
four character codes from four characters.

 Notice that the RFR prefix stands for Riff Read, and the RFW prefix stands for
Riff Write. An underscore character ``\verb+_+'' may be prepended to show that 
the corresponding function is internal to the library. Such functions should not
be accessed by the calling program under normal circumstances.

\subsubsection{RFRclose\_riff\_file()}
\begin{verbatim}
int
RFRclose_riff_file(riff_handle *handle);
\end{verbatim}

When you are done reading the file you can close it (and free the data structures
associated with the handle) by calling RFRclose\_riff\_file(). The handle will
no longer be accessible after that. 


\subsection{Reading blocks}

To read a block you need very few functions. The most important are:
\begin{verbatim}
int RFRget_block_info(riff_handle *handle, 
      uint32_t *type,
      uint32_t *size,
      uint32_t *list_type);
int RFRget_chunk(riff_handle *handle, 
      unsigned char *buffer);
int RFRskip_chunk(riff_handle *handle);
\end{verbatim}

\subsubsection{RFRget\_block\_info()}
The most important function is perhaps RFRget\_block\_info(). You need to call
RFRget\_block\_info() before every action that you plan to take, unless, of
course if you \emph{know} that this action is applicable under the current
circumstances. 

RFRget\_block\_info() returns the size of the current block and also copies
relevant information to the provided variables. The variable \verb+type+
contains the type of the next block which is either \verb+LIST+ or the
chunk type. The variable \verb+size+ contains the size of the next block.
The variable \verb+list_type+ refers to the list type of the current block,
\emph{if} it is a list. If the current block is not a list then \verb+list_type+
refers to the type of the list that contains it. 

% -- Done!
% \comment{IMPLEMENT LIST INFORMATION FOR CHUNKS!}

The function will not fail if pointers to type, size and list\_type are not
provided. It will return the current block size only.

\subsubsection{RFRget\_chunk()}
This function copies the data from the current chunk to the provided buffer.
This function will report an error if you do not provide an initialized buffer.

Please remember that \emph{no buffer overflow  checking is done} and it is
therefore possible that your program will perform a segment violation if 
the buffer is not at least of size equal to the current chunk's size.
To avoid such errors you should either use a sufficiently 
large buffer or ensure at all times that the current buffer is larger than
the size reported by RFRget\_block\_info for the current chunk.

The return value is the number of bytes copied.

\subsubsection{RFRskip\_chunk()}
This function proceeds to the next chunk without handling the current one.
It is assumed that the current block is always a chunk. In other words,
RFRskip\_chunk() will never skip a list. You may use this function to
skip chunks that you do not wish to process.

Return values are 0 for success, 1 if the current block is a list and $<0$
on various failures.

\subsection{Reading lists}
Several functions are useful when reading lists:
\begin{verbatim}
int RFRenter_list(riff_handle *handle);
int RFRskip_list(riff_handle *handle);
int RFRgoto_list_end(riff_handle *handle);
int RFRgoto_list_start(riff_handle *handle);
\end{verbatim}

\subsubsection{RFRenter\_list()}
This function allows you to proceed inside the list. It is applicable
only if the current block is a list. The alternative
function RFRskip\_list(), that is also applicable if the current block
is a list, will skip the list header and its contents.

The library does keep track of nested lists (lists inside other lists),
up to a maximum depth of 1023 lists. This function does some internal
book-keeping to ensure that the library is always accurately aware of
the list status. 

The return value is 0 on success, and, as usual $<0$ on various failures.

\subsubsection{RFRskip\_list()}
This function is applicable if the current block is a list. Contrary to
the RFRenter\_list() function it will skip the list header and the list
contents.   

The return value is 0 on success, and, as usual $<0$ on various failures.

\subsubsection{RFRgoto\_list\_end() and RFRgoto\_list\_start()}
These functions jump out of the list and move either to its end or to its
start. The effect is actually that of \emph{exiting} the list without 
producing a special list end block. The function RFRgoto\_list\_end()
should be treated as a way of skipping a list when inside it, its effect
is the same as RFRskip\_list() when at the start of the list. Note that
these functions are applicable only if the current block is a list.

The return value is 0 on success, and, as usual $<0$ on various failures.

\subsubsection{List end special blocks}
In order to keep your program aware of the current list status the library
will inform you by means of special ``phoney'' block information whenever
a list boundary is crossed. These blocks do \emph{not} exist in the actual
file, nor are they part of the RIFF specification. They are produced
by \libriff{}. 

% \comment{ IMPLEMENT block\_size == list\_type!}

The form of a list\_end block is:
\begin{verbatim}
/* This is a list */
block_type     = opLIST; 

/* This is the true type of the list that just ended! */
block_size     = list_type;
 
/* This list appears to be of type ``list'' */
block_subtype  = opLIST; 

/* Check for a list_end block */
RFRget_block_info(handle, type, size, list_type);

if (type==opLIST) {
     if (list_type==opLIST) 
        /* This is a special list end block. */
        /* The list of type ``size'' just ended. */
     else 
        /* This is a normal list */
} else
     /* This is a chunk */
\end{verbatim}

\subsection{Seeking}
Many functions are devoted to seeking. Most probably you won't
need every one of them. As a matter of fact, I strongly suggest
that you avoid them, with the exception of RFRfind\_chunk() and
RFReof(). Improper use of seeking can disrupt the internal data
structures and make the library fail. Great care must be taken 
not to exit or enter any lists without calling the appropriate list
functions. 

\begin{verbatim}
int
RFRseek_forward_to_chunk_type(riff_handle *input,uint32_t type);
int
RFRseek_forward_to_list_type(riff_handle *input,uint32_t type);
int
RFRfind_chunk(riff_handle *input, uint32_t type);

int
RFRseek_to_position(riff_handle *input, long file_position);
int
RFRsave_position(riff_handle *input);
int
RFRrestore_position(riff_handle *input);

long
RFRget_position(riff_handle *input);
int
RFRrewind(riff_handle *handle);

int
RFReof(riff_handle *input);
\end{verbatim}

\subsubsection{RFRseek\_forward\_to\_chunk\_type()}
This function and its brother RFRseek\_forward\_to\_list\_type()
will do exactly what their names imply. They will proceed in a forward
direction until they find a block of the appropriate type and they will
stop there, returning a value of 1. If the block cannot be found
these functions will reach the end of the file, entering and exiting
lists in the process. 

The most appropriate way of recovering from these functions is RFRrewind().
Simply seeking to original point in the file will \emph{NOT} do because 
the list keeping has changed and the library will fail. This is a direct
result of the inability to move backwards in RIFF files.
% (note: an indexing
% scheme maybe added on top of \libriff{} in the future)

\subsubsection{RFRfind\_chunk()}
This is the best function to use for searching. Contrary to the previous
functions, this function will \emph{NOT} exit
the current list. If it does not find a chunk of the required type it 
will stop at list end, file end or at the
start of a new list inside the current one, whichever occurs first. 
Upon completion a return value of 1 means that the chunk was found, 0
that the search had to stop. As always, $<0$ represents various
failures.

\subsubsection{RFRseek\_to\_position()}
This function will move the file pointer to the specified position. 
It will do this in a safe manner, respecting the file structure, and
is thus propably safer than \verb+fseek()+ especially when you do not
use the ``linear view'' feature (see below for this.)
Note that RFRseek\_to\_position() may need to
RFRrewind() the file and can be \emph{very} slow in degenerate cases. 
It is my understanding that in most cases it will work sufficiently
fast. 
%\emph{Please note that if the specified position is not the
%start of a block, bad things
%will happen. It is a critical assumption
%that the file pointer always lies at the start of a block.} 
As usual,
0 means success and $<0$ represents various failures.

\subsubsection{RFRsave\_position() and RFRrestore\_position()}
The function RFRsave\_position() saves the current file position in an internal
holder variable and RFRrestore\_position() will go back to that position.

\subsubsection{RFRget\_position()}
This function acts as a wrapper for ftell(). The return value, of type \verb+long+
is the file position.

\subsubsection{RFRrewind()}
This function will return the file pointer to position 12L, right \emph{after}
the RIFF file header, and properly initialize the internal data structures.
This function can be used to start scanning the file from the beginning.

\subsubsection{RFReof()}
This function returns 1 if the end of the file has been reached, 0 otherwise.

\subsubsection{Linear view}
%This is considered an {\bf experimental feature}. It may break some functions, but I think it is
%relatively safe. 

Normally RIFF files are implicitly ``hierarchical'' because of the
existence of lists. Every chunk belongs to the lowest level list that contains it, e.g.
in an avi file {\it strf\/} belongs to {\it strl\/} that is contained in {\it hdrl\/}. 
Linear view means that the file is seen in as a purely linear structure.
This has the advantage that seeking can be done efficiently using \verb+fseek()+ 
because we do not care about list boundaries
%\footnote{Naturally
%you must respect \emph{block} boundaries if you use {\tt fseek()} otherwise you will get garbage. \libriff{}
%cannot help in that case.}. 
The main disadvantage is that you may \emph{not} switch 
back to ``hierarchical'' view unless you \verb+RFRrewind()+ and you will \emph{not} get 
any list end notifications. You will normally
see all lists opening, but none will appear to close. To enable this feature set 
\verb+riff_handle->linear_view=1+ (0 by default).

An obvious use for this is in conjunction with the AVI file index that allows you to
safely move inside the {\it movi} list from keyframe to keyframe. After you parse the
AVI header you can enable linear view to allow fast seeking without the burdens of
list book-keeping.

Linear view does not work in riff writing mode. Its value is ignored. Also, please
note that the interface and/or behaviour may change slightly if any bugs are
discovered. 

\subsubsection*{Raw seeking}
You may use raw seeking either when the file does not keep hierarchical
list information (\emph{linear view} mentioned above) or when you can
guarantee that you will not cross list boundaries.

\begin{verbatim}
int _RFRraw_seek(riff_handle *, long);
\end{verbatim}

The above function works exactly like 
\verb+fseek()+ but also checks for some errors and calls \verb+_RFRget_next_info()+
afterwards to read block information. This means that the file position should
be a block start and also even, since the RIFF file format is explicitly
word aligned. If for some obscure reason you need to seek to a point that
is not a RIFF block start you may simply use plain-old \verb+fseek()+. 

The return value is normally 0 and will be \verb+<+0 if an error occurs.

\subsection{JUNK chunks and alignment}
Files of RIFF type are implicitly aligned on a 16-bit boundary at the
start of every block. However,
it is commonly preferable to align these files to some other boundary such
as 2048 bytes (data CD block size) by placing all or some of the blocks
in positions that are multiple of 2048.\footnote{Note that if you intend to
align a file so that it can be read quickly off a CD-ROM you will have to
align chunks to 2040 so that the chunk data starts at 2048, after the 8-byte header.}
This is achieved by using chunks
of type JUNK as padding. See the function RFWalign\_riff() for writing aligned
blocks.

The \libriff{} library has the ability to transparently \emph{skip} all
such chunks or treat them as normal chunks while reading. When in 
``skipping'' mode you will never encounter a chunk of type JUNK.
All are silently ignored. When in normal mode all JUNK chunks are 
properly reported and it is up to you to handle them or ignore them.

\subsubsection{RFRskip\_junk()}
You can use the function RFRskip\_junk() to
specify whether you would like the library to silently ignore all junk chunks.

\begin{verbatim}
int RFRskip_junk(riff_handle *handle, int skip_junk); 

RFRskip_junk(handle, 1);  /* Junk chunks will be ignored */  
RFRskip_junk(handle, 0);  /* Junk chunks will be processed */
\end{verbatim}

Junk skipping is off by default. Return value is equal to skip\_junk if successful,
or $<0$ on various failures.

\subsection{Opening files for writing}
 The library cannot handle a read/write mode. You can only read or write at
a file.
\subsubsection{RFWcreate\_riff\_handle()}
\begin{verbatim}
riff_handle * handle;

handle=RFWcreate_riff_handle(``filename'', uint32_t file_type);

if (handle==0)
	exit(-1); /* Could not open file */
else
	... /* File ready for writing. */
}
\end{verbatim}

\subsubsection{RFWclose\_riff()}
\begin{verbatim}
int
RFWclose_riff(riff_handle *handle);
\end{verbatim}
Closes the RIFF file and frees memory.
Returns -1 on failure, 0 on success. 

\subsubsection{RFWwrite\_chunk()}
\begin{verbatim}
int
RFWwrite_chunk(riff_handle *handle, 
               uint32_t type, 
               uint32_t length, 
               unsigned char *data)
\end{verbatim}

This function writes a chunk of type ``type'' with data of length ``length'' to
the associated file. The return value is equal to the number of bytes that actually
get written to the file. Note that this is not necessarily equal to ``length'', 
because it also includes at least 8 bytes for the header and, possibly, a single
byte for alignment purposes. A successful write must return length+8 or length+9.
Please remember that the data pointer must have been properly initialized.
If the function fails you will get a return value $<0$.

\subsubsection{RFWrewrite\_chunk()}
\begin{verbatim}
int
RFWrewrite_chunk(riff_handle *handle, 
	      uint32_t type, 
	      uint32_t length, 
	      unsigned char *data,
	      long file_pos)
\end{verbatim}
Rewrites a chunk at the specified position. This function's only difference with 
write\_chunk() is that it does not increase the file and 
also it does not do any alignment.
Returns number of bytes written, i.e. length plus 8 bytes for the header.
This is a potentially dangerous function. It can disrupt the RIFF file if 
not used with caution.

\subsubsection{RFWwrite\_raw\_chunk()}
\begin{verbatim}
int
RFWwrite_raw_chunk(riff_handle *handle, 
                   uint32_t length, 
                   unsigned char *data)
\end{verbatim}
Returns number of bytes written, normally equal to ``length''.
The chunks are written ``as-is'', and may not be legal if you do
not take care to supply an appropriate header. No alignment is done 
after the data, but this function will ensure that they start at an
aligned boundary. 

\subsubsection{RFWcreate\_list()}
\begin{verbatim}
int
RFWcreate_list(riff_handle *handle, uint32_t type)
\end{verbatim}
Creates a list of type ``type''. This function opens the list and reserves
space for its header. The list size is not written until you close the 
list with RFWclose\_list(). The return value is normally 12. Anything else
signifies an error.

\subsubsection{RFWclose\_list()}
\begin{verbatim}
int
RFWclose_list(riff_handle *handle)
\end{verbatim}
Closes the last list that was opened and writes its header properly. Note that
you can't have more than MAX\_LISTS (1024) lists open at once.

\subsubsection{RFWalign\_riff()}
\begin{verbatim}
int
RFWalign_riff(riff_handle *handle, int boundary)
\end{verbatim}
This function writes a JUNK chunk of the appropriate size
to align the file at the requested boundary. Note that the
requested boundary should be greater than 8 bytes, because
that is the size of the CHUNK header. You may not specify
an odd boundary, because RIFF files are word aligned.

\subsection{Miscellaneous functions}
\begin{verbatim}
char *    _RFWfcc2s(uint32_t fcc);
uint32_t  _RFWs2fcc(char c1, char c2, char c3, char c3);
\end{verbatim}
Several other functions of special, or limited interest exist.
I consider the above functions to be 
routinely quite useful. 
The first functions creates a 4 character, null
terminated string from a 32-bit unsigned int code  as in 
\verb+_RFWfcc2s(ophdrl)+, which should return ``hdrl''.
This function uses a static space (5 characters) to write
its output so every time you call it the previous result
gets overwritten. This is especially apparent if you
call it twice from \verb+printf()+ where you will only get
the result of the last call.
The second function builds an unsigned
32-bit int code from 4 characters as in \verb+_RFWfcc2s('s','t','r','l')+.


%\subsection{Is that all?}
%Several other functions also exist but are of limited interest. You are urged
%to read the header files to see if some of the more esoteric or trivial
%functions not presented here suit your needs. 

\section{Notes to developers}
If you are willing to work further \emph{on} this library, instead of working
\emph{with} it there are some minor points you should perhaps consider.

\subsection{Coding style}
 The coding style is as verbose as possible in order to ensure readability and
maintenability. I personally prefer the ``stroustrup'' coding style that emacs
offers. You can see for yourselves by examining the source code files. 

 According to the GNU coding standards all functions in the library are prefixed
to ensure that naming conflicts do not arise. Function names are prefixed with
{\bf RFW} if they are useful when writing and with {\bf RFR} if they are useful when 
reading. An underscore character ``\_'' is added to the front to denote functions
that are internal to the library and should not be invoked by the calling
program. 

 Comments should clearly describe what the code does and should mention important
events such as the opening of a file, allocation/deallocation of memory etc.
I consider a well written comment header before the function declaration
an element of good coding style. I tend to copy headers between .c and .h files 
to keep them up to date.

 Another point of interest is the examination of return values. I strongly 
suggest the checking of all return values, especially when calling functions
external to this library, e.g. C library functions. As a general rule, 
every function should verify its input parameters, the status of the internal
data structures and the return values of all called functions.

\subsubsection{Sanity checks and verbosity}
Debugging checks are done with the aid of the ``sanity.h'' facilities.
Two useful macros are provided : \texttt{SCHK(condition, action)} and 
\texttt{XCHK(condition, action)}. An SCHK is enabled whenever CHECKS is defined at
compiled. An XCHK is enabled whenever EXTENDED\_CHECKS is defined at compile time.
Generally speaking, SCHK is used for reasonable and often quick checks while XCHK
refers to more computationally expensive checks that are not likely to fail. The possible
actions are ACT\_FAIL, ACT\_WARN and ACT\_ERROR. I try to use ACT\_FAIL whenever something
has gone bad and cannot be corrected or understood. The default action, then is to 
exit() the program. This is done, for example, if the internal data structures
become corrupt, or if the library cannot allocate memory. An error that is
handled with ACT\_ERROR is usually less serious, possibly the result of bad calling
parameters. In that case, the action is to return(-1) from the current function.
The ACT\_WARN action means that only a message is printed. A message from SCHK() or
XCHK() will contain the appropriate filename, line number and function name to
guide you.\footnote{This is of course only supported if \_\_LINE\_\_, \_\_FILE\_\_ 
and \_\_FUNCTION\_\_ are properly defined by your compiler. Most probably they are.}

\subsection{Optimization}
 This library has \emph{not} been specially optimized for speed. The functions
that handle I/O cannot influence performance to a significant extent because
these functions do not contain nested loops or deep recursion and are mostly
bound by the time it takes to make the physical act of writing to the hard
drive/cd-rom/memory etc. Even when reading large files (\~{} 500MB) the CPU time
spent inside the library functions is negligible. Most functions are either
sufficiently fast or disk-bound.

 If you think that extra speed is required you may try to compile without
extended or standard sanity checks. See the file \texttt{``defines.h''} for this.
You may also omit debugging information (\texttt{-g} includes debugging 
information in GCC) and try some compiler optimization options such as
\texttt{-O3 -march=<your cpu>} for GCC.

 I do not intend to optimize this library for speed at the moment, if ever. 


\end{document}
